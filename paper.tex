\documentclass[parskip=half]{scrartcl}

\usepackage{graphicx}
\usepackage{booktabs}
\usepackage{amsmath}

\usepackage[
	left	= 2.5cm,
	right	= 2.5cm,
	top		= 2.5cm,
	bottom	= 3.0cm,
	includehead]{geometry}

\usepackage[headsepline,automark]{scrlayer-scrpage}
\newpairofpagestyles{standardheadings}{
	\clearpairofpagestyles
	\ohead{\pagemark}
	\ihead{\headmark}
	\KOMAoptions{headsepline=.5pt}
}
\pagestyle{standardheadings}

\title{make research}
% \title{\texttt{\$ make research -j16}}
\subtitle{an entry level tutorial on writing research papers using\\ GNU Make and CI/CD pipelines}
\author{Dmytro Strelnikov}

\begin{document}

\maketitle

\section{Problem}

In this dummy paper the finite elements method is applied to solve a boundary value problem on the cylinder $\Omega \times [0,T]$ where $\Omega = [0,1] \times [0,1]$. The problem consists of the homogeneous heat equation

\begin{equation} \label{eq:heat_eq}
	\alpha \frac{\partial \theta(x,t)}{\partial t} = \Delta \theta(x,t),
\end{equation}

with the following Neumann boundary conditions.

\begin{equation} \label{eq:bc}
	\frac{\partial \theta(x,t)}{\partial \vec{n}} = \left\{
		\begin{array}{ll}
			k (\theta(x,t) - \theta_\text{amb}), & \text{on}\ \Gamma_0, \\
			- u(t) g(x), & \text{on}\ \Gamma_1, \\
			0, & \text{on}\ \partial\Omega \setminus (\Gamma_0 \cup \Gamma_1).
		\end{array}
		\right.
\end{equation}

Here $\Gamma_0 = [0,1] \times \{0\}$, $\Gamma_1 = [0,1] \times \{1\}$. Functions $g(x_1, 1)$ and $u(t)$ are presented in Figure~\ref{fig:bc}.

\begin{figure} \label{fig:bc}
	\centering
	\includegraphics{plots/g.pdf}
	\includegraphics{plots/control.pdf}
	\caption{Boundary conditions applied in the paper.}
\end{figure}


\section{Solution}

The solution to the problem at selected time moments is presented in Figure~\ref{fig:solution}.

\begin{figure} \label{fig:solution}
	\centering
	\includegraphics{plots/solution.pdf}
	\caption{Solution to the equation at different time moments.}
\end{figure}

The corresponding maximal and minimal temperatures are presented in Table~\ref{tab:temperatures}.

\begin{table} \label{tab:temperatures}
	\centering
	\input{tables/temperatures.tex}
	\caption{Maximal and minimal temperatures at different time moments.}
\end{table}

\end{document}
 
