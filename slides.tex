\documentclass[
	9pt,
	hyperref = {unicode,pdfpagelabels=false},
	aspectratio = 43
	]{beamer}
\mode<presentation>

\usepackage{tikz-cd}
\usepackage{listings}
\usepackage{mdframed}

\usetheme{metropolis}
\metroset{
	titleformat = smallcaps,
	titleformat title = smallcaps,
	sectionpage = progressbar,
	progressbar = foot,
	block = fill
}

\usefonttheme{professionalfonts}
\setsansfont{IBM Plex Sans}

\title{make research}
% \title{\texttt{\$ make research -j16}}
\subtitle{an entry level tutorial on writing research papers using\\ GNU Make and CI/CD pipelines}

\author{Dmytro Strelnikov}

\institute{Technische Universität Chemnitz}

\date{\today}

\begin{document}
\maketitle
\begin{frame}{Let's write a paper}
	As an example I'm going to write a small paper describing a numerical solution to the heat equation.

	The paper will include:
	\begin{itemize}
		\item a plot of the boudary condition
		\item plots of the solution at different time points
		\item a table with numerical data about the solution
	\end{itemize}
\end{frame}

\begin{frame}{What can go wrong?}
	\textbf{Reproducibility}
	\begin{itemize}
		\item an interested reader won't be able to reproduce the result
		\item in a two months I won't be able to reproduce the result
		\item in a year I won't be able to even make my code running
	\end{itemize}

	\textbf{Consistency}
	\begin{itemize}
		\item I won't identify an error at the time it appeared
		\item I will adjust the problem formulation while writing and present inconsistent data in the final paper
	\end{itemize}

	\textbf{Workflow}
	\begin{itemize}
		\item I will run the computations and process the data manually
	\end{itemize}

	\textbf{Delivery}
	\begin{itemize}
		\item I will keep artifacts in the version controlled repository
		\item the artifacts will get out of sync with their sources
	\end{itemize}
\end{frame}

\begin{frame}{How to do it the right way?}
	Outline the design and let the computer work for you.
	The tools can be different, however the basic principles remain the same:
	\begin{itemize}
		\item understand your project dependencies, represent them as a DAG
		\item keep a single point of responsibility, don't repeat yourself
		\item make it fully reproducible, let the paper write itself
		\item automate the build process and test it after every change
		\item keep derivatives separate from the original work
	\end{itemize}
\end{frame}

\begin{frame}[fragile]{Dependency graph}
	As the first step we identify the artifacts and group them. We describe the dependencies between those groups as a DAG.

	\vspace{1em}

	\begin{tikzcd}[column sep=large]
		& & \text{plots} \ar[r, "\text{TeX}"] \ar[ddr]& \text{paper} \\
		\ar[r, "\text{compute}"] & \text{numericals}
			\ar[dr, "\text{tabulate}", bend right=20]
			\ar[ur, "\text{plot}", bend left=20] \\
		& & \text{tables} \ar[r] \ar[uur]& \text{slides} \\
	\end{tikzcd}

	This graph will serve as a template for the Makefile and the CI/CD pipeline description.
\end{frame}

\begin{frame}[fragile]{Makefile}
	GNU Make is a tool which controls the generation of artifacts (executables, pictures, PDF files), i.e. any non-source files of a project.

	Make gets its knowledge of how to build your project from a file called the Makefile, which lists each of the non-source files and how to compute it from other files.

	\begin{figure}
	\centering
	\lstinputlisting[language=make, firstline=25, lastline=29, frame=lines, boxpos=c]{Makefile}
	\caption{The basic entity of the Makefile.}
	\end{figure}
\end{frame}


\end{document}
